\subsection{Лекция 1}

\textit{Закон больших чисел}

Под законом больших чисел в широком смысле понимается общий принцип, согласно которому совокупное действие большого числа случайных факторов приводит к неслучайному результату.
При большом количестве случайных величин средний результат их совокупного влияния может быть предсказан с высокой степенью определенности.

Под законом больших чисел в узком смысле понимается ряд математических теорем, в каждой из которых для тех или иных условий устанавливается факт приближения средних характеристик большого числа испытаний к некоторым постоянным.

\begin{center}
    \textbf{Неравенство Маркова}
\end{center} 

Если случайная величина $X \ge 0$ и имеет конечное $M(x)$ , то $\forall A > 0$ справедливо неравенство $P(X>A) \le \frac{MX}{A}$

\textbf{Доказательство для дискретной случайной величины: }

$ MX=\sum \limits_{i=1}^{n}{x_i p_i} $ отбросим первые $k$ слагаемых: $ \sum \limits_{i=1}^{k}{x_i p_i} \ge 0 $; $ \sum \limits_{i=k+1}^{n}{x_i p_i} \le MX $ 
заменим $x_i$ на $A$ :
$ \sum \limits_{i=k+1}^{n}{A_i p_i} \le MX \Rightarrow \sum \limits_{i=k+1}^{n}{p_i} \le \frac{MX}{A} \Rightarrow P(X>A) \le \frac{MX}{A} $
$P(x \le A) \ge 1 - \frac{MX}{A}$

\begin{center}
    \textbf{Неравенство Чебышева}
\end{center} 
