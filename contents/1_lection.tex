\subsection{Лекция 1}

\textit{Закон больших чисел}

Под законом больших чисел в широком смысле понимается общий принцип, согласно которому совокупное действие большого числа случайных факторов приводит к неслучайному результату.
При большом количестве случайных величин средний результат их совокупного влияния может быть предсказан с высокой степенью определенности.

Под законом больших чисел в узком смысле понимается ряд математических теорем, в каждой из которых для тех или иных условий устанавливается факт приближения средних характеристик большого числа испытаний к некоторым постоянным.

\begin{center}
    \textbf{Неравенство Маркова}
\end{center} 

Если случайная величина $X \ge 0$ и имеет конечное $M(x)$ , то $\forall A > 0$ справедливо неравенство $P(X>A) \le \frac{MX}{A}$

\textbf{Доказательство для дискретной случайной величины: }

$ MX=\sum \limits_{i=1}^{n}{x_i p_i} $ отбросим первые $k$ слагаемых: $ \sum \limits_{i=1}^{k}{x_i p_i} \ge 0 $; $ \sum \limits_{i=k+1}^{n}{x_i p_i} \le MX $ 

заменим $x_i$ на $A$ :

$ \sum \limits_{i=k+1}^{n}{A_i p_i} \le MX \Rightarrow \sum \limits_{i=k+1}^{n}{p_i} \le \frac{MX}{A} \Rightarrow P(X>A) \le \frac{MX}{A} $\

Следствие: $P(x \le A) \ge 1 - \frac{MX}{A}$

\begin{center}
    \textbf{Неравенство Чебышева}
\end{center} 

Для любой случайной величины, имеющей конечной математическое ожидания и дисперсию справедливо, что $P(|X-MX|>\varepsilon) \le \frac{DX}{\varepsilon^2}$

\textbf{Доказательство: }

$Y=(X-MX)^2;A=\varepsilon^2$ запишем неравенство Маркова $P((X-MX)^2 > \varepsilon) \le \frac{M((X-MX)^2)}{\varepsilon^2} = \frac{DX}{\varepsilon^2} \Rightarrow$ извлекаем корень и получаем неравенство.

Следствие: $P(|X-MX| \le \varepsilon) \ge 1 - \frac{DX}{\varepsilon^2}$

\begin{center}
    \textbf{Теорема Чебышева}
\end{center} 

Если случайные величины $X_1,X_2 ... X_n$ независимые и имеют конечные дисперсии, ограниченные одной и той же константой $C$, то при неограниченном увеличении $n$ среднее арифметическое этих случайных величин сходится по вероятности 
к среднему арифметическому математических ожиданий.
$$\lim \limits_{n \to \infty} {P(|\frac{X_1+X_2+ ... +X_n}{n} - \frac{MX_1+MX_2+ ...+MX_n}{n}| < \varepsilon)=1}; \forall \varepsilon > 0$$
$ \frac{X_1+X_2+ ... +X_n}{n} \stackrel{P}{\xrightarrow[n\to\infty]{}} \frac{MX_1+MX_2+ ...+MX_n}{n} $ сходится по вероятности.

\textbf{Доказательство: }

$X=\frac{X_1+X_2+ ... +X_n}{n}; MX=\frac{MX_1+MX_2+ ... +MX_n}{n} \\
DX=\frac{1}{n^2}(DX_1+DX_2+ ... +DX_n) \le \frac{nC}{n^2} = \frac{C}{n}$ \\
запишем неравенство Чебышева для $X$ \\
$P(|X-MX| \le \varepsilon) \ge 1 - \frac{DX}{\varepsilon^2} \ge 1 - \frac{C}{n \varepsilon^2}$ \\
$\lim \limits_{n \to \infty} {P(|\frac{X_1+X_2+ ... +X_n}{n} - \frac{MX_1+MX_2+ ...+MX_n}{n}| < \varepsilon) \ge 1-\frac{C}{n \varepsilon^2}}$, т.к $1-\frac{C}{n \varepsilon^2} \to 1$, то \\
$\lim \limits_{n \to \infty} {P(|\frac{X_1+X_2+ ... +X_n}{n} - \frac{MX_1+MX_2+ ...+MX_n}{n}| < \varepsilon)=1}$

\textbf {Следствия}

\begin{enumerate}
  \item Если все случайные величины $X_i$ имеют одно и тоже математическое ожидание $X_i=a$ \\ 
    $\lim \limits_{x \to \infty} {P(|\frac{X_1+X_2+ ... +X_n}{n} - a| < \varepsilon)=1}$
  \item \textbf {Теорема Бернулли} \\
    Частость события в $n$ повторных независимых испытаниях, в каждом из которых она может произойти с вероятностью $P$ при неограниченном увеличении числа
    испытаний стремится к вероятности этого события\\
    \textbf{Доказательство: }\\
    $\lim \limits_{n \to \infty} {P(|\frac{X_1+X_2+ ... +X_n}{n} - \frac{MX_1+MX_2+ ...+MX_n}{n}| < \varepsilon)=1}$ \\
    $\frac{X_1+X_2+ ... +X_n}{n}$ - частость; $\frac{MX_1+MX_2+ ...+MX_n}{n}$ - вероятность \\
    $x_i=1$, если событие проявилось в $i$-ом испытании \\
    $x_i=0$, иначе \\
    Частость: $\frac{m}{n}$, где $m$- кол-во встречаний; $n$ - общее кол-во \\
    $MX_i=p \Rightarrow \lim \limits_{n \to +\infty}{P(|\frac{m}{n}|<\varepsilon)}=1 $
\end{enumerate}

\begin{center}
    \textbf{Центральная предельная теорема}
\end{center} 

Рассмотренный нами закон больших чисел устанавливает факт приближения средних значений большего числа случайных величин к определенным постоянным, а центральная предельная теорема 
устанавливает распределение, которое проявляется при совокупном влиянии большего числа случайных факторов.

Центральная предельная теорема - это группа теорем, посвященных установлению условий, при которых при сложении большого числа случайных величин возникает нормальный закон распределения.

\textbf{Центральная предельная теорема (формулировка Ляпунова)}

Если $x_1,x_2 ... x_n$ - независимые случайные величины, у каждой из которых существуют конечные $MX_i,DX_i$; $MX_i=a_i; DX_i= \sigma_i^2$, кроме того у каждой случайной величины существует абсолютный центральный момент 3 порядка 
$M(|X_i-a_i|^3)=m_i(*)$, кроме того $\lim \limits_{n \to + \infty} {\frac{\sum \limits_{i=1}^{n}{m_i}}{(\sum \limits_{i=1}^{n}{\sigma_i^2})^{3/2}}}=0(**)$, то закон распределения суммы случайных величин $X_1+X_2+ .. +X_n$ с ростом $n$ 
неограниченно приближается к нормальному закону распределения с $M=\sum \limits_{i=1}^{n}{a_i}$ и $D=\sum \limits_{i=1}^{n}{\sigma_i^2}$

В практическом смысле требования $(*),(**)$ означают, что в сумме $X_1+X_2+ .. +X_n$ не должно быть слагаемых, влияние которых подавляюще велико по сравнению с влиянием остальных, а кроме того не должно быть большого числа слагаемых,
суммарное влияние которых на итоговые суммы незначительно.

Следствие: локальная и интегральная теоремы Муавра-Лапласа

\begin{center}
    \textbf{Центральная предельная теорема}
\end{center} 

\textbf{Задачи математической статистики}

\begin{enumerate}
 \item Указать способы сбора и группировки статистических сведений, полученных в результате наблюдений или статистических экспериментов
 \item Разработать методы анализа статистических данных в зависимости от цели исследований
       \begin{itemize}
        \item Оценка неизвестной вероятности события, оценка неизвестной функции распределения, оценка параметров извеестного распределения,оценка зависимости одной случайной величины от другой
        \item Проверка статистических гипотез о видах распределения и значения параметров
       \end{itemize}
\end{enumerate}

\textbf{Def1} Выборочной совокупность (выборкой) называют совокупность случайно отобранных объектов.

\textbf{Def2} Генеральной совокупностью называют полную совокупность объектов,из которых производится выборка.

\textbf{Def3} Выборку называют повторной, если каждый отобранный элемент перед отбором следующего элемента возвращается в генеральную совокупность.

\textbf{Def4} Выборку называют бесповторной, если каждый отобранный элемент больше не возвращается в генеральную совокупность.

\textbf{Def5} Выборка называется репрезентативной, если она правильно отражает пропорции генеральной совокупности.

\textbf{Способы отбора}

\begin{enumerate}
 \item Простой случайный отбор (объекты отбираются по одному из всей генеральной совокупности) (повторный или бесповторный)
 \item Типический отбор (объекты выбираются не из всей генеральной совокупности, а из каждой типической части)
 \item Механический отбор (генеральную совокупность делят на столько групп, сколько элементов должно войти в выборку, потом из каждой группы берут по одному)
 \item Серийный отбор (отбор, при котором объекты отбирают из генеральной совокупности не по одному, а сериями)
\end{enumerate}
